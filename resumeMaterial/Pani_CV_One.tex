%%%%%%%%%%%%%%%%%%%%%%%%%%%%%%%%%%%%%%%%%
% Plasmati Graduate CV
% LaTeX Template
% Version 1.0 (24/3/13)
%
% This template has been downloaded from:
% http://www.LaTeXTemplates.com
%
% Original author:
% Alessandro Plasmati (alessandro.plasmati@gmail.com)
%
% License:
% CC BY-NC-SA 3.0 (http://creativecommons.org/licenses/by-nc-sa/3.0/)
%
% Important note:
% This template needs to be compiled with XeLaTeX.
% The main document font is called Fontin and can be downloaded for free
% from here: http://www.exljbris.com/fontin.html
%
%%%%%%%%%%%%%%%%%%%%%%%%%%%%%%%%%%%%%%%%%

%----------------------------------------------------------------------------------------
%	PACKAGES AND OTHER DOCUMENT CONFIGURATIONS
%----------------------------------------------------------------------------------------

\documentclass[a4paper,10pt]{extarticle} % Default font size and paper size

\usepackage{fontspec} % For loading fonts
\defaultfontfeatures{Mapping=tex-text}
\setmainfont[SmallCapsFont = Fontin SmallCaps]{Fontin} % Main document font
\fontspec{[FontAwesome.otf]}


\usepackage{xunicode,xltxtra,url,parskip} % Formatting packages

\usepackage[usenames,dvipsnames]{xcolor} % Required for specifying custom colors

%\usepackage[big]{layaureo} % Margin formatting of the A4 page, an alternative to layaureo can be 
%\usepackage{fullpage}
\usepackage{geometry}
\geometry{a4paper,margin=0.75cm}
%\geometry{a4paper,left=20mm, top=20mm}
%To reduce the height of the top margin uncomment: \addtolength{\voffset}{-1.3cm}

\usepackage{hyperref} % Required for adding links	and customizing them
%\definecolor{linkcolour}{rgb}{0,0.2,0.6} % Link color
\definecolor{linkcolour}{rgb}{0.3,0.3,0.3} % Link color
\hypersetup{colorlinks,breaklinks,urlcolor=linkcolour,linkcolor=linkcolour} % Set link colors throughout the document

\usepackage{titlesec} % Used to customize the \section command
\titleformat{\section}{\large\scshape\raggedright}{}{0em}{}[\titlerule] % Text formatting of sections
\titlespacing{\section}{0pt}{0pt}{0pt} % Spacing around sections

\usepackage{multicol}
\setlength{\columnsep}{0cm}

\usepackage{textcomp}

\usepackage{fontawesome}

\def\arraystretch{0.85}

\begin{document}
	
	\pagestyle{empty} % Removes page numbering
	
	\font\fb=''[cmr10]'' % Change the font of the \LaTeX command under the skills section

%----------------------------------------------------------------------------------------
%	NAME AND CONTACT INFORMATION
%----------------------------------------------------------------------------------------
\begin{multicols}{3}
	% \par{\centering\normalsize {\textsc{Undergraduate student at Indian Institute of Technology, Kharagpur}}\par}\normalsize
	% \par{\centering\normalsize {\textsc{Department of Computer Science and Engineering}}\par}\normalsize
	%\par{{\begin{center}Dual Degree, \emph{Computer Science and Engineering}\end{center}}}
	\normalsize  \faGlobe\ {\href{https://sandeepkumarpani888.github.io/blog/}{sandeepkumarpani888.me}}\\
	\normalsize \faGithub\ {\href{https://github.com/sandeepkumarpani888}{sandeepkumarpani888}}\\
	\normalsize  \faLinkedinSquare\ {\href{https://in.linkedin.com/in/sandeep-pani-3a87117b
			}{sandeep-kumar-pani}}\\
	\columnbreak
	\normalsize\par{\centering{\huge R \textsc{Sandeep}}\par} % Your name
	\par{\centering\normalsize {\textsc{A-213, LBS Hall of Residence, IIT Kharagpur, West Bengal, India - 721302}}\hfill\par}
	\columnbreak
	\raggedright\hfill\normalsize \faEnvelope\ {\href{mailto:sandeepkumarpani888@gmail.com}{sandeepkumarpani888@gmail.com}}\\
	\raggedright\hfill{\faPhone\ +91-9874262606}
\end{multicols}


%\section{Research Interests}

%- Software Engineering\hfill\\
%- Algorithms\hfill

%----------------------------------------------------------------------------------------
%	EDUCATION
%----------------------------------------------------------------------------------------

\section{Education}

\begin{tabular}{r|p{17cm}}	
2013-2018 & B.Tech and M.Tech (Dual Degree) in \textsc{Computer Science and Engineering}\\
\textsc{(Expected)}&\textbf{Indian Institute of Technology}, Kharagpur\\
&\textbf{Coursework: }{Programming and Data Structures, Discrete Structures, Algorithms-II, Switching Circuits, Operating Systems, Computer Networks, Machine Learning, Image Processing, Algorithms-I, Software Engineering, Compilers, Database Management Systems, Information Retrieval, Advanced Graph Theory}
\end{tabular}

%----------------------------------------------------------------------------------------
%	SKILLS 
%----------------------------------------------------------------------------------------


\section{Technical Skills}

\begin{tabular}{r|p{17cm}}
	\textsc{Programming} & {\itshape{Proficient in}} C, C++ and \itshape{Familiar with} Javascript, Python, Java, \verb!C#! \\
	\textsc{Libraries/Frameworks} & Node.js, AngularJS, Express, Socket.io\\
	\textsc{Databases} & MySQL, MongoDB\\
	\textsc{Systems/Platforms} & {\itshape{Familiar with }}Git, AWS (EC2), Android
\end{tabular}

%----------------------------------------------------------------------------------------
%	Software Projects
%----------------------------------------------------------------------------------------

\section{Experience}

\begin{tabular}{r|p{17cm}}

\textsc{Jun 2016} & \textbf{Software Engineer Intern}\hfill\textbf{Microsoft India Development Centre, Hyderabad}\\
\textsc{May 2016}& \footnotesize{- Part of the Azure Site Recovery team, which aims to provide consistent apps, workloads and data during planned and unplanned downtime or disasters, and to recover to normal working conditions (Disaster Recovery).}\\
& \footnotesize{- As part of the team, I was responsible for helping provide Disaster-Recovery support for AWS cloud environment from the ground up. Worked extensively with Amazon EC2 instances and Azure, and was able to integrate the project on their current stack. Implemented several proof of concepts and investigation phases to help keep the system in a consistent state.}\\
& \footnotesize{- Have been offered a Full-Time position on the basis of the work done. }\\
\end{tabular}

\section{Academic Projects}

\begin{tabular}{r|p{17cm}}
%\emph{Current} & 1\textsuperscript{st} year Analyst at \textsc{Lehman Brothers}, London \\

\textsc{Apr 2016} & \textbf{Data extraction from biomedical literature for automating systematic reviews} \\
% & \textbf{B.Tech Project}\\
% & \textbf{Guide: }\textmd{\href{http://cse.iitkgp.ac.in/~pawang/}{Professor Pawan Goyal}}\\
& \footnotesize{- Worked on feature detection of a particular class of text (specifically, inclusion and exclusion criteria for patients) from a huge collection of biomedical literature using NLP Techniques with high precision and recall.}\\
% & \footnotesize{- Methods used include Support Vector Classifier (Scikit-learn), Latent Dirichlet Allocation (LDA) and Weighted Keyword Matching.}\\
\multicolumn{2}{c}{} \\

\textsc{Apr 2016} & \textbf{Selene} \textsc{(A community based music-recommendation engine.)} \\
% & \textbf{Guide: }\textmd{\href{http://cse.iitkgp.ac.in/~pabitra/}{Professor Pabitra Mitra}}\\
& \footnotesize{- An android app that serves as a social music recommendation engine using Facebook's graph API, YouTube's API and Musixmatch's API giving more priority to music listened to by your close friends.}\\
\multicolumn{2}{c}{} \\

\textsc{Mar 2016} & \textbf{\href{https://github.com/sandeepkumarpani888/dbmsass3}{Studious - Course Management System}}\\
% & \textbf{Guide: }\textmd{\href{http://cse.iitkgp.ac.in/~pabitra/}{Professor Pabitra Mitra}}\\
& \footnotesize{- Built a complete course management system that supported authentication \& authorization, User Access Control for 4 different types of users, real-time messaging with notifications (using socket.io), calendar support and all major features one can expect from a CMS including faculty management, course progression, self-evaluated tests, etc.}\\
% & \footnotesize{- The complete workflow was built using the MEAN stack. Twitter Bootstrap was utilised for making the site fully responsive.}\\
\multicolumn{2}{c}{} \\

\textsc{Apr 2016} & \textbf{Retrieving salient sentences from Reddit AMAs} \\
% \textsc{Mar 2016} & \textbf{Guide: }\textmd{\href{http://cse.iitkgp.ac.in/~pawang/}{Professor Pawan Goyal}}\\
& \footnotesize{- Built a web-based summariser that provides summaries from /r/iAMA with abilities to choose any AMA from a list or through instant search (implemented with Angular autocomplete.)}\\
% & \footnotesize{- After obtaining data from the web crawler, LDA model training was done on the entire dataset and categorised using link flairs on Reddit. k-mean clustering was then used to cluster the questions and answers and summarised using lexrank and summpy. Concept tagging was then done using Alchemy API on each cluster.}\\
\multicolumn{2}{c}{} \\

\textsc{Apr 2015} & \textbf{Hall Management System}\\
% & \textbf{Guide: }\textmd{\href{http://www.iitkgp.ac.in/fac-profiles/showprofile.php?empcode=SSmUZ}{Professor Partha Pratim Das}}\\
& \footnotesize{- Built a hall management system using MySQL and Java frontend which automates the process of managing various aspects of hall related activites like hall and room allocation, staff management and general overview of expenditure to name a few.}\\
\multicolumn{2}{c}{} \\
\end{tabular}

%------------------------------------------------
%	Projects and Competitions
%------------------------------------------------
\section{Hackathons \& Workshops}

\begin{tabular}{r|p{17cm}}


\textsc{Apr 2016} & \textbf{\href{https://github.com/sandeepkumarpani888/Opensoft-2016}{Data Extractor from 2D plots}}\hfill\textbf{OpenSoft 2016}\\
& \footnotesize{- Built a graph extractor that detects multi-variable graphs in any given PDF and tabulates them autonomously taking into consideration features like axis values, scale, legend}\\
% & \footnotesize{- Was primarily responsible for detecting and scaling the ticks from the colour segregated image and scale it appropriately using the axis values and generate the plot. Was also solely responsible for generating the table structure using Python libraries and to fully build a working GUI for the application on Java Swing.}\\
\multicolumn{2}{c}{} \\

\textsc{Jun 2015} & \textbf{\href{https://github.com/sandeepkumarpani888/apollo-api}{AppoloX}}\hfill\textbf{Angel Hacks 2015}\\
& \footnotesize{- Took part in \textbf{Angel Hacks Delhi}, a 24 hour hackathon, where my team built AppoloX, a context sensitive app which understands the context of the text and recommends activities accordingly. We used node.js, brain.js and the Android Platform.}\\
\multicolumn{2}{c}{} \\

\textsc{Mar 2015} & \textbf{Code.Fun.Do} \\
& \footnotesize{- Built an app to help people with their therapy plans using \textbf{KINECT} to check the movement of joints and voice prompts to correct the movements during the \textbf{Microsoft Code.Fun.Do Hackathon}}\\
\multicolumn{2}{c}{} \\
\end{tabular}

%----------------------------------------------------------------------------------------
%	SPORT PROGRAMMING
%----------------------------------------------------------------------------------------

\section{Sport Programming}
- Google APAC-2016 Round-B \textbf{Rank-157}\\
- ACM-ICPC Onsite Regionals \textbf{Rank-63} (Team Event)\\
- Member of Team yeah\_u\_wish, one among 25 teams all over india selected for KTJ coding event OVERNITE and secured \textbf{17th position} there.\\
- Regular on codeforces as {\href{http://codeforces.com/profile/renegade_warrior}{renegade\_warrior}}\\

%----------------------------------------------------------------------------------------
%	SKILLS 
%----------------------------------------------------------------------------------------

% \section{Computer Skills}

% \begin{tabular}{r|p{15cm}}
% \textsc{Programming Languages} & C, C++, Java, Python, AngularJS\\
% \multicolumn{2}{c}{} \\
% \textsc{Software Packages} & Visual Studio, Eclipse, Codeblocks\\
% \multicolumn{2}{c}{} \\
% \textsc{Frameworks} & MEAN stack\\
% \end{tabular}

%----------------------------------------------------------------------------------------
%	COURSEWORK
%----------------------------------------------------------------------------------------

% \section{Coursework
% \hfill\small\textsc{(T)heory and (L)aboratory}}

% \begin{multicols}{2}

% - Programming and Data Structures (T/L) \\
% - Algorithms-I (T/L) \\
% - Discrete Structures \\
% - Software Engineering (T/L) \\
% - Formal Languages and Automata Theory \\
% - Algorithms - II \\
% - Compilers (T/L) \\
% - Computer Organisation and Architecture (T/L) \\
% - Matrix Algebra \\
% - Database Management Systems (T/L) \\
% - Operating Systems (T/L) \\
% - Computer Networks (T/L) \\
% - Information Retrieval\\
% - Speech and Natural Language Processing
% \end{multicols}

%----------------------------------------------------------------------------------------
%	SCHOLASTIC ACHIEVEMENTS
%----------------------------------------------------------------------------------------

% \section{Scholastic Achievements}

% \hspace*{0.5cm}- Secured All India Rank - 479 in \textbf{IIT JEE- ADVANCED 2013}.\\
% \hspace*{0.5cm}- Secured All India Rank - 416 in \textbf{IIT JEE-MAINS 2013}.\\
% \hspace*{0.5cm}- Secured the \textbf{KVPY Scholarship}.\\
% \hspace*{0.5cm}- Qualified for the second level in \textbf{Indian National Physics Olympiad}.\\

%----------------------------------------------------------------------------------------
%	EXTRA CURRICULAR ACTIVITIES
%----------------------------------------------------------------------------------------

% \section{Extra Curricular Activities}

% \hspace*{0.5cm}- Member of National Social Service\\
% \hspace*{0.5cm}- Kshitij volunteer during the main event KTJ 2014\\

%----------------------------------------------------------------------------------------

%\newpage
%----------------------------------------------------------------------------------------

\end{document}
